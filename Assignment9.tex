% Options for packages loaded elsewhere
\PassOptionsToPackage{unicode}{hyperref}
\PassOptionsToPackage{hyphens}{url}
%
\documentclass[
]{article}
\usepackage{amsmath,amssymb}
\usepackage{iftex}
\ifPDFTeX
  \usepackage[T1]{fontenc}
  \usepackage[utf8]{inputenc}
  \usepackage{textcomp} % provide euro and other symbols
\else % if luatex or xetex
  \usepackage{unicode-math} % this also loads fontspec
  \defaultfontfeatures{Scale=MatchLowercase}
  \defaultfontfeatures[\rmfamily]{Ligatures=TeX,Scale=1}
\fi
\usepackage{lmodern}
\ifPDFTeX\else
  % xetex/luatex font selection
\fi
% Use upquote if available, for straight quotes in verbatim environments
\IfFileExists{upquote.sty}{\usepackage{upquote}}{}
\IfFileExists{microtype.sty}{% use microtype if available
  \usepackage[]{microtype}
  \UseMicrotypeSet[protrusion]{basicmath} % disable protrusion for tt fonts
}{}
\makeatletter
\@ifundefined{KOMAClassName}{% if non-KOMA class
  \IfFileExists{parskip.sty}{%
    \usepackage{parskip}
  }{% else
    \setlength{\parindent}{0pt}
    \setlength{\parskip}{6pt plus 2pt minus 1pt}}
}{% if KOMA class
  \KOMAoptions{parskip=half}}
\makeatother
\usepackage{xcolor}
\usepackage[margin=1in]{geometry}
\usepackage{color}
\usepackage{fancyvrb}
\newcommand{\VerbBar}{|}
\newcommand{\VERB}{\Verb[commandchars=\\\{\}]}
\DefineVerbatimEnvironment{Highlighting}{Verbatim}{commandchars=\\\{\}}
% Add ',fontsize=\small' for more characters per line
\usepackage{framed}
\definecolor{shadecolor}{RGB}{248,248,248}
\newenvironment{Shaded}{\begin{snugshade}}{\end{snugshade}}
\newcommand{\AlertTok}[1]{\textcolor[rgb]{0.94,0.16,0.16}{#1}}
\newcommand{\AnnotationTok}[1]{\textcolor[rgb]{0.56,0.35,0.01}{\textbf{\textit{#1}}}}
\newcommand{\AttributeTok}[1]{\textcolor[rgb]{0.13,0.29,0.53}{#1}}
\newcommand{\BaseNTok}[1]{\textcolor[rgb]{0.00,0.00,0.81}{#1}}
\newcommand{\BuiltInTok}[1]{#1}
\newcommand{\CharTok}[1]{\textcolor[rgb]{0.31,0.60,0.02}{#1}}
\newcommand{\CommentTok}[1]{\textcolor[rgb]{0.56,0.35,0.01}{\textit{#1}}}
\newcommand{\CommentVarTok}[1]{\textcolor[rgb]{0.56,0.35,0.01}{\textbf{\textit{#1}}}}
\newcommand{\ConstantTok}[1]{\textcolor[rgb]{0.56,0.35,0.01}{#1}}
\newcommand{\ControlFlowTok}[1]{\textcolor[rgb]{0.13,0.29,0.53}{\textbf{#1}}}
\newcommand{\DataTypeTok}[1]{\textcolor[rgb]{0.13,0.29,0.53}{#1}}
\newcommand{\DecValTok}[1]{\textcolor[rgb]{0.00,0.00,0.81}{#1}}
\newcommand{\DocumentationTok}[1]{\textcolor[rgb]{0.56,0.35,0.01}{\textbf{\textit{#1}}}}
\newcommand{\ErrorTok}[1]{\textcolor[rgb]{0.64,0.00,0.00}{\textbf{#1}}}
\newcommand{\ExtensionTok}[1]{#1}
\newcommand{\FloatTok}[1]{\textcolor[rgb]{0.00,0.00,0.81}{#1}}
\newcommand{\FunctionTok}[1]{\textcolor[rgb]{0.13,0.29,0.53}{\textbf{#1}}}
\newcommand{\ImportTok}[1]{#1}
\newcommand{\InformationTok}[1]{\textcolor[rgb]{0.56,0.35,0.01}{\textbf{\textit{#1}}}}
\newcommand{\KeywordTok}[1]{\textcolor[rgb]{0.13,0.29,0.53}{\textbf{#1}}}
\newcommand{\NormalTok}[1]{#1}
\newcommand{\OperatorTok}[1]{\textcolor[rgb]{0.81,0.36,0.00}{\textbf{#1}}}
\newcommand{\OtherTok}[1]{\textcolor[rgb]{0.56,0.35,0.01}{#1}}
\newcommand{\PreprocessorTok}[1]{\textcolor[rgb]{0.56,0.35,0.01}{\textit{#1}}}
\newcommand{\RegionMarkerTok}[1]{#1}
\newcommand{\SpecialCharTok}[1]{\textcolor[rgb]{0.81,0.36,0.00}{\textbf{#1}}}
\newcommand{\SpecialStringTok}[1]{\textcolor[rgb]{0.31,0.60,0.02}{#1}}
\newcommand{\StringTok}[1]{\textcolor[rgb]{0.31,0.60,0.02}{#1}}
\newcommand{\VariableTok}[1]{\textcolor[rgb]{0.00,0.00,0.00}{#1}}
\newcommand{\VerbatimStringTok}[1]{\textcolor[rgb]{0.31,0.60,0.02}{#1}}
\newcommand{\WarningTok}[1]{\textcolor[rgb]{0.56,0.35,0.01}{\textbf{\textit{#1}}}}
\usepackage{graphicx}
\makeatletter
\def\maxwidth{\ifdim\Gin@nat@width>\linewidth\linewidth\else\Gin@nat@width\fi}
\def\maxheight{\ifdim\Gin@nat@height>\textheight\textheight\else\Gin@nat@height\fi}
\makeatother
% Scale images if necessary, so that they will not overflow the page
% margins by default, and it is still possible to overwrite the defaults
% using explicit options in \includegraphics[width, height, ...]{}
\setkeys{Gin}{width=\maxwidth,height=\maxheight,keepaspectratio}
% Set default figure placement to htbp
\makeatletter
\def\fps@figure{htbp}
\makeatother
\setlength{\emergencystretch}{3em} % prevent overfull lines
\providecommand{\tightlist}{%
  \setlength{\itemsep}{0pt}\setlength{\parskip}{0pt}}
\setcounter{secnumdepth}{-\maxdimen} % remove section numbering
\ifLuaTeX
  \usepackage{selnolig}  % disable illegal ligatures
\fi
\IfFileExists{bookmark.sty}{\usepackage{bookmark}}{\usepackage{hyperref}}
\IfFileExists{xurl.sty}{\usepackage{xurl}}{} % add URL line breaks if available
\urlstyle{same}
\hypersetup{
  pdftitle={Activity \#10: Reproducible RMDs},
  pdfauthor={Maeric Barrows},
  hidelinks,
  pdfcreator={LaTeX via pandoc}}

\title{Activity \#10: Reproducible RMDs}
\author{Maeric Barrows}
\date{2023-11-21}

\begin{document}
\maketitle

\hypertarget{prologue-introducing-the-assignment}{%
\subsection{Prologue: Introducing the
Assignment}\label{prologue-introducing-the-assignment}}

This R Markdown file is an extension of Activity 9: A First RMD File.
This project was made in conjunction with GitHub. The link to the GitHub
repository is provided here:
\url{https://github.com/maeric75/STAT184_Activity_10}

(Note for Prof.~Hatfield: The screenshot of my file directory is
provided at the bottom of the assignment, above the code appendix. Also,
the name for this file says ``Assigment 9.'' This is Activity 10, I just
forgot to make a copy of it before editing this file and I just ran with
it.)

(Note for me: I (Maeric as of 11/21/2023) did not change any of the
material from Activity 9. I just made the RMD file more accessible for
future me. I hope I helped you.)

\hypertarget{section-1-collatz-conjecture}{%
\subsection{Section 1: Collatz
Conjecture}\label{section-1-collatz-conjecture}}

The Collatz conjecture, a famous function in mathematics, tests whether
every integer eventually becomes the number 1 by running the number
through two arithmetic operations. These are the operations present in
the conjecture:

\begin{itemize}
\tightlist
\item
  If n (the integer being tested) is even: n = n / 2
\item
  If n is odd: n = 3n + 1
\item
  If n = 1: Stop the conjecture
\end{itemize}

In Assignment 3 of STAT 184, Prof.~Hatfield asked the students to find
the ``stopping times'' for the first 10,000 integers greater than 0, and
create a histogram of these stopping times. A stopping time is how many
times a number goes through the Collatz conjecture before it becomes the
number 1. To find this, we must first create code which will properly
evaluate the Collatz conjecture. Then, we will use the sapply function
so that the Collatz conjecture runs from integers 1 to 10,000. After
this, we will turn our Collatz data into a data frame.

Then, we will use ggplot2 to create a histogram using the data frame:

\includegraphics{Assignment9_files/figure-latex/unnamed-chunk-5-1.pdf}

To make a conclusion about the data, we can observe the ``flow'' of the
histogram. The bars close to the stopping number 0 are very short,
followed by taller and taller bars until we get to just above the
stopping number 50, where the bars start to get shorter again. In
between stopping numbers 100 and 150, the bars generally increase in
height for a bit, but go back to steadily decreasing in height after.
The height of a bar corresponds to how many times a stopping number
occurs. Our observations tell us that the very smallest and very largest
stopping numbers do not occur very often, and that a significant amount
of integers between 1 and 10,000 stop after about 50 recursions of the
Collatz conjecture. The amount of integers that stop after about 125
recursions is also greater than the numbers which stop after 100 or 150
recursions.

\hypertarget{section-2-the-price-of-diamonds}{%
\subsection{Section 2: The Price of
Diamonds}\label{section-2-the-price-of-diamonds}}

In addition to creating data visualizations, the ggplot2 package
features a data set about diamonds. The data set features data for
53,940 diamonds, measuring them based on many variables, like carat
(weight), cut quality, clarity, color, width (x), height (y), and depth
(z), among others. All of these attributes contribute to a diamond's
price. We can see the relationship between these variables and price by
creating data visualizations and tables.

First, we'll create a data visualization comparing the carat of a
diamond and its price. Since there are so many diamonds in the data set,
and we are looking for a general trend when comparing carat and price, a
scatter plot is a good way to demonstrate this data.

\includegraphics{Assignment9_files/figure-latex/unnamed-chunk-6-1.pdf}

This visualization shows that as the carat, or weight, of a diamond
increases, the price also generally increases. This comparison is
simple. If we want to add more variables to this graph, we can split all
of the diamonds by their clarity. The scatter plot below still compares
carat and price, but also compares those two variables to a diamond's
clarity:

\includegraphics{Assignment9_files/figure-latex/unnamed-chunk-7-1.pdf}

This graph also shows that price increases as carat increases. This
visualization adds another comparison point with the clarity of the
diamond. We can see that the plots that represent lower clarity diamonds
(grouped in the top row of scatter plots) reach the highest prices
further to the right than the plots with higher clarity (grouped in the
middle and bottom row). This proves that as a diamond's clarity
increases, its price generally increases as well.

Visualizations are not the only way to demonstrate data and make
conclusions about what results in a diamond's price. We can also create
summary tables. Here is code that will create a summary table of the cut
of a diamond and its price:

\begin{verbatim}
##         cut price_min price_Q1 price_median price_Q3 price_max price_mean
## 1      Fair       337  2050.25       3282.0  5205.50     18574   4358.758
## 2      Good       327  1145.00       3050.5  5028.00     18788   3928.864
## 3 Very Good       336   912.00       2648.0  5372.75     18818   3981.760
## 4   Premium       326  1046.00       3185.0  6296.00     18823   4584.258
## 5     Ideal       326   878.00       1810.0  4678.50     18806   3457.542
##   price_stdev count
## 1    3560.387  1610
## 2    3681.590  4906
## 3    3935.862 12082
## 4    4349.205 13791
## 5    3808.401 21551
\end{verbatim}

This summary table provides a lot of statistics, some of which are
confusing to decipher. Before creating the table, I assumed that
diamonds with an ``Ideal'' cut would have a higher price than those with
lower quality cuts, like ``Fair'' or ``Good.'' In reality, the data does
not support that hypothesis. ``Ideal'' diamonds actually have the lowest
minimum, first quartile, median, third quartile, and average price, the
exact opposite of what I would expected. The lowest quality cut,
``Fair,'' has the highest minimum, first quartile, and median price,
while ``Premium'' has the highest third quartile and mean price.
``Ideal'' is very close to first in maximum price, but is still slightly
behind ``Very Good'' and ``Premium.'' This does not necessarily mean
that ``Ideal'' diamonds absolutely have the lowest prices. Since there
are much more ``Ideal'' diamonds than any other in this study, as shown
by the ``count'' column, it is possible that the data of the ``Ideal''
column is more accurate than the other cut types. It is difficult to
make a concrete conclusion, but using this data, we can infer that cut
quality has little to no effect on the price of a diamond.

\hypertarget{section-3-course-takeaways-so-far}{%
\subsection{Section 3: Course Takeaways (So
Far)}\label{section-3-course-takeaways-so-far}}

For Section 3 of this RMD assignment, Prof.~Hatfield asked us to reflect
on what we have learned in the course. Something that I have always
thought, since the beginning of the course, is that R is an incredibly
overwhelming software to use. There are so many libraries of functions
that could be used to solve any data-related problem you could think of.
I am sure that I have not touched more than 2\% of what R has to offer
in this course, and the 2\% I have seen is still filled with so many
options for what to do that I find myself extremely overwhelmed in every
assignment. Thus, with there being a lot to learn about R, I have
already learned a lot. What follows is my attempt to bring up some
things not touched on in this assignment already.

\textbf{Learned Thing 1. PCIP}

When first introduced to PCIP, or ``Plan, Code, Improve, Polish,'' I was
doubtful that it would help me code in R. From the beginning, I liked to
throw code at the wall and see what stuck. This worked in Python
classes, but I found R to be an entirely different beast. As I
mentioned, everything was so overwhelming for me. To fix this, I needed
to slow my process down, and organize it a little better. PCIP has
gotten me out of many ruts in this class, and I am thankful that I
started taking the process seriously.

\textbf{Learned Thing 2. Wrangling and Cleaning}

When I first heard of data wrangling, I was excited, because I hoped it
would pertain to a hobby of mine. I create my own Top 40 hit songs chart
every week, in the style of the Billboard Hot 100, which involves taking
data and reinterpret it. Making one of these charts typically takes
hours, and I thought that data wrangling would automate the process a
little. Unfortunately, I have not learned of a way to speed up my
personal process yet, as the data I use comes in the form of an image,
but if I can find data in other forms, like spreadsheets, I will be able
to use my new-found data wrangling and cleaning skills to automate my
Top 40 chart more than I already have. Below is the result of data
wrangling I did from a previous assignment. I was very proud of figuring
this one out.

\begin{verbatim}
## New names:
## * `` -> `...1`
## * `` -> `...3`
## * `` -> `...4`
## * `` -> `...6`
## * `` -> `...7`
## * `` -> `...9`
## * `` -> `...10`
## * `` -> `...12`
## * `` -> `...13`
## * `` -> `...15`
## * `` -> `...16`
\end{verbatim}

\begin{verbatim}
## # A tibble: 192 x 4
##    `Pay Grade` `Marital Status`        Sex    Count
##    <chr>       <chr>                   <chr>  <dbl>
##  1 E-1         Single Without Children Male    9456
##  2 E-1         Single Without Children Female  1309
##  3 E-1         Single With Children    Male     365
##  4 E-1         Single With Children    Female    80
##  5 E-1         Joint Service Marriage  Male      40
##  6 E-1         Joint Service Marriage  Female    38
##  7 E-1         Civilian Marriage       Male    2579
##  8 E-1         Civilian Marriage       Female   358
##  9 E-2         Single Without Children Male   21600
## 10 E-2         Single Without Children Female  3324
## # i 182 more rows
\end{verbatim}

\textbf{Learned Thing 3. Perseverance}

This will sound corny, but this class has taught me to persevere and
work hard. In high school, I was used to things coming easy to me. I
would pick up concepts quickly, and classes felt like a breeze. Here,
things changed. I often felt like I was making no progress and that I
was hitting a wall in my learning. Most assignments in this class take
hours for me to complete, with more than a few taking days, something I
was not accustomed to. Taking this class, and many others in my first
semester, has been humbling. Even though I do not enjoy the stressful
process of struggling, the feeling of pride at the end for completing a
tough assignment is ultimately worth it. It is difficult to convey my
level of perseverance in a data visualization, but this class has
certainly made me realize how much it rewards me, so a visualization for
my attitude about perseverance might look like a line graph trending
upwards.

\hypertarget{epilogue-screenshot-for-prof.-hatfield}{%
\subsection{Epilogue: Screenshot for
Prof.~Hatfield}\label{epilogue-screenshot-for-prof.-hatfield}}

\begin{figure}
\centering
\includegraphics{"~/RMarkdown Files/Activity10Screenshot.png"}
\caption{Screenshot of GitHub Desktop file directory}
\end{figure}

\newpage

\hypertarget{code-appendix}{%
\section{Code Appendix}\label{code-appendix}}

\begin{Shaded}
\begin{Highlighting}[]
\NormalTok{knitr}\SpecialCharTok{::}\NormalTok{opts\_chunk}\SpecialCharTok{$}\FunctionTok{set}\NormalTok{(}\AttributeTok{echo =} \ConstantTok{FALSE}\NormalTok{)}
\CommentTok{\# Call the groundhog package.}
\FunctionTok{library}\NormalTok{(groundhog)}
\CommentTok{\# Load packages with groundhog to improve stability.}
\NormalTok{pkgs }\OtherTok{\textless{}{-}} \FunctionTok{c}\NormalTok{(}\StringTok{"ggplot2"}\NormalTok{, }\StringTok{"dplyr"}\NormalTok{, }\StringTok{"readxl"}\NormalTok{, }\StringTok{"tidyr"}\NormalTok{)}
\FunctionTok{groundhog.library}\NormalTok{(pkgs, }\StringTok{\textquotesingle{}2023{-}11{-}05\textquotesingle{}}\NormalTok{)}
\CommentTok{\# Install tinytex for pdf knitting}
\FunctionTok{library}\NormalTok{(tinytex)}
\DocumentationTok{\#\#\# Calculate the Collatz conjecture and return the stopping number}
\NormalTok{calculate\_collatz }\OtherTok{\textless{}{-}} \ControlFlowTok{function}\NormalTok{(number, }\AttributeTok{count =} \DecValTok{0}\NormalTok{)\{}
  \ControlFlowTok{if}\NormalTok{(number }\SpecialCharTok{==} \DecValTok{1}\NormalTok{)\{}
    \FunctionTok{return}\NormalTok{(count)}
\NormalTok{  \}}\ControlFlowTok{else} \ControlFlowTok{if}\NormalTok{(number }\SpecialCharTok{\%\%} \DecValTok{2} \SpecialCharTok{==} \DecValTok{0}\NormalTok{)\{}
\NormalTok{    new\_number }\OtherTok{\textless{}{-}}\NormalTok{ number }\SpecialCharTok{/} \DecValTok{2}
    \FunctionTok{calculate\_collatz}\NormalTok{(}\AttributeTok{number =}\NormalTok{ new\_number, }\AttributeTok{count =}\NormalTok{ count }\SpecialCharTok{+} \DecValTok{1}\NormalTok{)}
\NormalTok{  \}}\ControlFlowTok{else}\NormalTok{\{}
\NormalTok{    new\_number }\OtherTok{\textless{}{-}} \DecValTok{3}\SpecialCharTok{*}\NormalTok{number }\SpecialCharTok{+} \DecValTok{1}
    \FunctionTok{calculate\_collatz}\NormalTok{(}\AttributeTok{number =}\NormalTok{ new\_number, }\AttributeTok{count =}\NormalTok{ count }\SpecialCharTok{+} \DecValTok{1}\NormalTok{)}
\NormalTok{  \}}
\NormalTok{\}}
\DocumentationTok{\#\#\# Collects the stopping numbers when calculate\_collatz is run using numbers 1 to 10,000.}
\NormalTok{sapply\_collatz }\OtherTok{\textless{}{-}} \FunctionTok{sapply}\NormalTok{(}
  \AttributeTok{X =} \FunctionTok{seq}\NormalTok{(}\DecValTok{1}\NormalTok{,}\DecValTok{10000}\NormalTok{,}\DecValTok{1}\NormalTok{),}
  \AttributeTok{FUN =}\NormalTok{ calculate\_collatz}
\NormalTok{)}
\DocumentationTok{\#\#\# Creates a data frame to store the values of sapply\_collatz.}
\NormalTok{collatz\_dataframe }\OtherTok{\textless{}{-}} \FunctionTok{as.data.frame}\NormalTok{(}
  \AttributeTok{x =}\NormalTok{ sapply\_collatz}
\NormalTok{)}
\DocumentationTok{\#\#\# Creates a histogram using collatz\_dataframe.}
\FunctionTok{ggplot}\NormalTok{(}
  \AttributeTok{data =}\NormalTok{ collatz\_dataframe,}
  \AttributeTok{mapping =} \FunctionTok{aes}\NormalTok{(}\AttributeTok{x =} \StringTok{\textasciigrave{}}\AttributeTok{sapply\_collatz}\StringTok{\textasciigrave{}}\NormalTok{)}
\NormalTok{) }\SpecialCharTok{+}
  \FunctionTok{geom\_histogram}\NormalTok{(}
    \AttributeTok{bins =} \DecValTok{50}\NormalTok{,}
    \AttributeTok{fill =} \StringTok{"cyan"}\NormalTok{,}
    \AttributeTok{color =} \StringTok{"black"}
\NormalTok{  ) }\SpecialCharTok{+}
  \FunctionTok{labs}\NormalTok{(}
    \AttributeTok{x =} \StringTok{"Stopping Numbers"}\NormalTok{,}
    \AttributeTok{y =} \StringTok{"Count"}\NormalTok{,}
    \AttributeTok{title =} \StringTok{"Histogram of Stopping Numbers"}\NormalTok{,}
    \AttributeTok{subtitle =} \StringTok{"Collatz Conjecture Input from 1 to 10,000"}
\NormalTok{  ) }\SpecialCharTok{+}
  \FunctionTok{theme\_linedraw}\NormalTok{() }\SpecialCharTok{+}
  \FunctionTok{theme}\NormalTok{(}
    \AttributeTok{plot.title =} \FunctionTok{element\_text}\NormalTok{(}\AttributeTok{size =}\NormalTok{ 16L,}
                              \AttributeTok{face =} \StringTok{"bold"}\NormalTok{,}
                              \AttributeTok{hjust =} \FloatTok{0.5}\NormalTok{),}
    \AttributeTok{plot.subtitle =} \FunctionTok{element\_text}\NormalTok{(}\AttributeTok{size =}\NormalTok{ 12L,}
                                 \AttributeTok{hjust =} \FloatTok{0.5}\NormalTok{),}
    \AttributeTok{axis.title.x =} \FunctionTok{element\_text}\NormalTok{(}\AttributeTok{size =}\NormalTok{ 12L),}
    \AttributeTok{axis.title.y =} \FunctionTok{element\_text}\NormalTok{(}\AttributeTok{size =}\NormalTok{ 12L)}
\NormalTok{  )}
\DocumentationTok{\#\#\# Creates a scatter plot to compare carat and price of a diamond.}
\FunctionTok{ggplot}\NormalTok{(}
  \AttributeTok{data =}\NormalTok{ diamonds,}
  \AttributeTok{mapping =} \FunctionTok{aes}\NormalTok{(}\AttributeTok{x =}\NormalTok{ carat,}
                \AttributeTok{y =}\NormalTok{ price)}
\NormalTok{) }\SpecialCharTok{+}
  \FunctionTok{geom\_point}\NormalTok{(}\AttributeTok{shape =} \StringTok{"circle"}\NormalTok{,}
             \AttributeTok{size =} \FloatTok{1.5}\NormalTok{,}
             \AttributeTok{color =} \StringTok{"blue"}\NormalTok{) }\SpecialCharTok{+}
  \FunctionTok{labs}\NormalTok{(}
    \AttributeTok{x =} \StringTok{"Carat"}\NormalTok{,}
    \AttributeTok{y =} \StringTok{"Price"}\NormalTok{,}
    \AttributeTok{title =} \StringTok{"Carat vs. Price of a Diamond"}
\NormalTok{  ) }\SpecialCharTok{+}
  \FunctionTok{theme\_linedraw}\NormalTok{() }\SpecialCharTok{+}
  \FunctionTok{theme}\NormalTok{(}
    \AttributeTok{plot.title =} \FunctionTok{element\_text}\NormalTok{(}\AttributeTok{size =}\NormalTok{ 16L,}
                              \AttributeTok{face =} \StringTok{"bold"}\NormalTok{,}
                              \AttributeTok{hjust =} \FloatTok{0.5}\NormalTok{),}
    \AttributeTok{axis.title.x =} \FunctionTok{element\_text}\NormalTok{(}\AttributeTok{size =}\NormalTok{ 12L),}
    \AttributeTok{axis.title.y =} \FunctionTok{element\_text}\NormalTok{(}\AttributeTok{size =}\NormalTok{ 12L)}
\NormalTok{    )}
\DocumentationTok{\#\#\# Creates a graph that groups all diamonds by clarity, and compares carat and price.}
\FunctionTok{ggplot}\NormalTok{(}
  \AttributeTok{data =}\NormalTok{ diamonds,}
  \AttributeTok{mapping =} \FunctionTok{aes}\NormalTok{(}\AttributeTok{x =}\NormalTok{ carat,}
                \AttributeTok{y =}\NormalTok{ price)}
\NormalTok{) }\SpecialCharTok{+}
  \FunctionTok{geom\_point}\NormalTok{(}\AttributeTok{shape =} \StringTok{"circle"}\NormalTok{,}
             \AttributeTok{size =} \FloatTok{1.5}\NormalTok{,}
             \AttributeTok{color =} \StringTok{"blue"}\NormalTok{) }\SpecialCharTok{+}
  \FunctionTok{labs}\NormalTok{(}
    \AttributeTok{x =} \StringTok{"Carat"}\NormalTok{,}
    \AttributeTok{y =} \StringTok{"Price"}\NormalTok{,}
    \AttributeTok{title =} \StringTok{"Carat vs. Price as Separated by Clarity"}
\NormalTok{  ) }\SpecialCharTok{+}
  \FunctionTok{theme\_linedraw}\NormalTok{() }\SpecialCharTok{+}
  \FunctionTok{theme}\NormalTok{(}
    \AttributeTok{plot.title =} \FunctionTok{element\_text}\NormalTok{(}\AttributeTok{size =}\NormalTok{ 16L,}
                              \AttributeTok{face =} \StringTok{"bold"}\NormalTok{,}
                              \AttributeTok{hjust =} \FloatTok{0.5}\NormalTok{),}
    \AttributeTok{axis.title.x =} \FunctionTok{element\_text}\NormalTok{(}\AttributeTok{size =}\NormalTok{ 12L),}
    \AttributeTok{axis.title.y =} \FunctionTok{element\_text}\NormalTok{(}\AttributeTok{size =}\NormalTok{ 12L)}
\NormalTok{    ) }\SpecialCharTok{+}
  \FunctionTok{facet\_wrap}\NormalTok{(}\FunctionTok{vars}\NormalTok{(clarity))}
\DocumentationTok{\#\#\# Create a summary table using cut and price.}
\CommentTok{\# Load data}
\FunctionTok{data}\NormalTok{(diamonds)}

\CommentTok{\# Summarize data}
\NormalTok{cut\_price\_summary }\OtherTok{\textless{}{-}}\NormalTok{ diamonds }\SpecialCharTok{\%\textgreater{}\%}
  \FunctionTok{group\_by}\NormalTok{(cut) }\SpecialCharTok{\%\textgreater{}\%}
  \FunctionTok{select}\NormalTok{(cut, price) }\SpecialCharTok{\%\textgreater{}\%}
  \FunctionTok{summarize}\NormalTok{(}
    \FunctionTok{across}\NormalTok{(}
      \AttributeTok{.cols =} \FunctionTok{where}\NormalTok{(is.numeric),}
      \AttributeTok{.fns =} \FunctionTok{list}\NormalTok{(}
        \AttributeTok{min =} \SpecialCharTok{\textasciitilde{}}\FunctionTok{min}\NormalTok{(price, }\AttributeTok{na.rm =} \ConstantTok{TRUE}\NormalTok{),}
        \AttributeTok{Q1 =} \SpecialCharTok{\textasciitilde{}}\FunctionTok{quantile}\NormalTok{(price, }\AttributeTok{probs =} \FloatTok{0.25}\NormalTok{, }\AttributeTok{na.rm =} \ConstantTok{TRUE}\NormalTok{),}
        \AttributeTok{median =} \SpecialCharTok{\textasciitilde{}}\FunctionTok{median}\NormalTok{(price, }\AttributeTok{na.rm =} \ConstantTok{TRUE}\NormalTok{),}
        \AttributeTok{Q3 =} \SpecialCharTok{\textasciitilde{}}\FunctionTok{quantile}\NormalTok{(price, }\AttributeTok{probs =} \FloatTok{0.75}\NormalTok{, }\AttributeTok{na.rm =} \ConstantTok{TRUE}\NormalTok{),}
        \AttributeTok{max =} \SpecialCharTok{\textasciitilde{}}\FunctionTok{max}\NormalTok{(price, }\AttributeTok{na.rm =} \ConstantTok{TRUE}\NormalTok{),}
        \AttributeTok{mean =} \SpecialCharTok{\textasciitilde{}}\FunctionTok{mean}\NormalTok{(price, }\AttributeTok{na.rm =} \ConstantTok{TRUE}\NormalTok{),}
        \AttributeTok{stdev =} \SpecialCharTok{\textasciitilde{}}\FunctionTok{sd}\NormalTok{(price, }\AttributeTok{na.rm =} \ConstantTok{TRUE}\NormalTok{)}
\NormalTok{      )}
\NormalTok{    ),}
    \AttributeTok{count =} \FunctionTok{n}\NormalTok{()}
\NormalTok{  )}

\CommentTok{\# Convert to table}
\NormalTok{cut\_price\_table }\OtherTok{\textless{}{-}} \FunctionTok{as.data.frame}\NormalTok{(cut\_price\_summary)}

\NormalTok{cut\_price\_table}
\DocumentationTok{\#\#\# Wrangles the army data from previous assignments.}
\DocumentationTok{\#\# Getting Stuff}
\CommentTok{\# Set working directory}
\FunctionTok{setwd}\NormalTok{(}\StringTok{"C:/Users/Maeric/OneDrive/Documents"}\NormalTok{)}

\CommentTok{\# Read in data from Excel file using a relative file path}
\NormalTok{army\_data }\OtherTok{\textless{}{-}} \FunctionTok{read\_xlsx}\NormalTok{(}
  \AttributeTok{path =} \StringTok{"\textasciitilde{}/RMarkdown Files/Army\_MaritalStatus.xlsx"}\NormalTok{,}
  \AttributeTok{range =} \StringTok{"Sheet1!B8:Q37"}
\NormalTok{)}

\DocumentationTok{\#\# Wrangling}
\CommentTok{\# Removing unnecessary rows and columns}
\NormalTok{army\_data }\OtherTok{\textless{}{-}}\NormalTok{ army\_data }\SpecialCharTok{\%\textgreater{}\%}
  \FunctionTok{select}\NormalTok{(}\FunctionTok{c}\NormalTok{(}\DecValTok{1}\SpecialCharTok{:}\DecValTok{3}\NormalTok{, }\DecValTok{5}\NormalTok{,}\DecValTok{6}\NormalTok{, }\DecValTok{8}\NormalTok{,}\DecValTok{9}\NormalTok{, }\DecValTok{11}\NormalTok{,}\DecValTok{12}\NormalTok{))}\SpecialCharTok{\%\textgreater{}\%}
  \FunctionTok{slice}\NormalTok{(}\FunctionTok{c}\NormalTok{(}\DecValTok{1}\SpecialCharTok{:}\DecValTok{10}\NormalTok{, }\DecValTok{12}\SpecialCharTok{:}\DecValTok{21}\NormalTok{, }\DecValTok{23}\SpecialCharTok{:}\DecValTok{27}\NormalTok{))}
\FunctionTok{View}\NormalTok{(army\_data)}

\CommentTok{\# Creating the group case data frame}
\NormalTok{wrangled\_army\_data }\OtherTok{\textless{}{-}}\NormalTok{ army\_data }\SpecialCharTok{\%\textgreater{}\%}
  \FunctionTok{rename}\NormalTok{(}\FunctionTok{c}\NormalTok{(}\StringTok{"Pay Grade"} \OtherTok{=} \DecValTok{1}\NormalTok{,}
           \StringTok{"Single Without Children\_Male"} \OtherTok{=} \DecValTok{2}\NormalTok{,}
           \StringTok{"Single Without Children\_Female"} \OtherTok{=} \DecValTok{3}\NormalTok{,}
           \StringTok{"Single With Children\_Male"} \OtherTok{=} \DecValTok{4}\NormalTok{,}
           \StringTok{"Single With Children\_Female"} \OtherTok{=} \DecValTok{5}\NormalTok{,}
           \StringTok{"Joint Service Marriage\_Male"} \OtherTok{=} \DecValTok{6}\NormalTok{,}
           \StringTok{"Joint Service Marriage\_Female"} \OtherTok{=} \DecValTok{7}\NormalTok{,}
           \StringTok{"Civilian Marriage\_Male"} \OtherTok{=} \DecValTok{8}\NormalTok{,}
           \StringTok{"Civilian Marriage\_Female"} \OtherTok{=} \DecValTok{9}\NormalTok{)) }\SpecialCharTok{\%\textgreater{}\%}
  \FunctionTok{slice}\NormalTok{(}\SpecialCharTok{{-}}\DecValTok{1}\NormalTok{) }\SpecialCharTok{\%\textgreater{}\%}
  \FunctionTok{pivot\_longer}\NormalTok{(}
    \AttributeTok{cols =} \SpecialCharTok{!}\StringTok{"Pay Grade"}\NormalTok{,}
    \AttributeTok{names\_to =} \StringTok{"marital\_status"}\NormalTok{,}
    \AttributeTok{values\_to =} \StringTok{"Count"}
\NormalTok{  ) }\SpecialCharTok{\%\textgreater{}\%}
\NormalTok{  tidyr}\SpecialCharTok{::}\FunctionTok{separate\_wider\_delim}\NormalTok{(}
    \AttributeTok{cols =}\NormalTok{ marital\_status,}
    \AttributeTok{delim =} \StringTok{"\_"}\NormalTok{,}
    \AttributeTok{names =} \FunctionTok{c}\NormalTok{(}\StringTok{"Marital Status"}\NormalTok{, }\StringTok{"Sex"}\NormalTok{)}
\NormalTok{  ) }\SpecialCharTok{\%\textgreater{}\%}
  \FunctionTok{mutate}\NormalTok{(}
    \AttributeTok{Count =} \FunctionTok{as.numeric}\NormalTok{(Count)}
\NormalTok{  )}

\NormalTok{wrangled\_army\_data}
\end{Highlighting}
\end{Shaded}


\end{document}
